% ----------------------------------------------------------------------
%  *** DOCUMENT INFO ***
%
% Title: IHM-Labo-02
% Matière: IHM
% Auteur: Sébastien Richoz & Damien Rochat
% ----------------------------------------------------------------------
\documentclass[11pt, a4paper, french]{article}

% -----------------------------------------------------------------------
% *** PACKAGES ***
% -----------------------------------------------------------------------
\usepackage[utf8]{inputenc} % encodage accent
\usepackage[T1]{fontenc} % encodage accent
\usepackage{ae, pslatex} % Joli output en PDF
\usepackage[french]{babel} % Renomme les noms des chapitres, dates, ...
\usepackage{listings} % pour highlighter du code mySQL
\usepackage{listingsutf8}
\usepackage{xcolor}
\usepackage{graphicx} % Pour gérer des images
\usepackage{fancyhdr} % En-têtes améliorés
\usepackage{multicol}
\usepackage{vmargin}
\usepackage{etoolbox} % structure if else simplifiée

% -----------------------------------------------------------------------
%  *** CONFIGS ***
% -----------------------------------------------------------------------
\definecolor{LightGray}{gray}{0.9}
\definecolor{codegreen}{rgb}{0,0.6,0}
\definecolor{codegray}{rgb}{0.5,0.5,0.5}
\definecolor{codepurple}{rgb}{0.58,0,0.82}
\graphicspath{ {images/} } % Chemin des images

\usepackage[pdftex, % metadata fichiers pdf
			pdftitle={IHM Labo 02},
			pdfauthor={Sébastien Richoz, Damien Rochat},
			pdfsubject={Rapport IHM Labo 02}]{hyperref}

\setlength\parindent{0pt} % pas d'indentation de la première ligne des paragraphes


% -----------------------------------------------------------------------
%  *** MARGES ***
% -----------------------------------------------------------------------
%\addtolength{\textwidth}{170pt} \addtolength{\hoffset}{-85pt}
%\addtolength{\textheight}{150pt} \addtolength{\voffset}{-75pt}

\setpapersize{A4}
\setmarginsrb{40pt}{24pt}{40pt}{24pt}{14pt}{24pt}{14pt}{24pt}
%\setmarginsrb{a}{b}{c}{d}{e}{f}{g}{h}
%
% a => left margin
% b => top margin
% c => right margin
% d => bottom margin
% e => header height
% f => distance between header and text
% g => footer height
% h => distance between bottom page and bottom footer


% -----------------------------------------------------------------------
%  *** EN-TETE ET PIED DE PAGE ***
% -----------------------------------------------------------------------
\pagestyle{fancy}

\renewcommand{\sectionmark}[1]{\markright{\thesection\ #1}} % Noms des sections en minuscule
\fancyhf{}  												% Supprime les entetes et pieds existants

\fancyhead[L]{IHM-Labo-02}  % titre
\fancyfoot[C]{Page \thepage}  							% N° de la page
\fancyfoot[L]{Sébastien Richoz \& Damien Rochat}		% pieds de page	
\fancyfoot[R]{Novembre 2016}							% date
\renewcommand{\headrulewidth}{0.5pt}
\renewcommand{\footrulewidth}{0.5pt}

% -----------------------------------------------------------------------
%  *** TITRE ***
% -----------------------------------------------------------------------
\title{IHM Labo 02}
\author{Sébastien Richoz, Damien Rochat}
\date{Novembre 2016}

% ****************************************************************************************************************
% --- DOCUMENT PRINCIPAL ---
% ****************************************************************************************************************
\begin{document}
	\maketitle

	\part*{Interface}
		 L'interface de l'application permet à l'utilisateur d'aller de haut en bas, comme le fait naturellement une personne lorsqu'elle regarde quelque chose. \\
		 
		 L'utilisateur a tout d'abord la possibilité de sélectionner une vidéo, en passant par le bouton \textbf{Open video} ou en cliquant dans la grande zone centrale, comme les personnes en ont l'habitude lorsqu'une icône semblable (représentant un upload) est visible. Une fois la vidéo sélectionnée, ses informations sont affichées et la vidéo peut être lue. Ceci simplifie grandement la manière de trouver les passages recherchés. \\
		 
		 Les composants suivants permettent de définir la coupure que l'ont souhaite. Deux manières sont proposées à l'utilisateur, soit en passant par le "double slider" ou alors en passant par les boutons \textbf{plus} et \textbf{moins} respectifs pour modifier le début ou la fin de la vidéo. La visualisation de la vidéo est automatiquement mise à jour afin que l'utilisateur puisse être le plus précis possible. La durée de la vidéo résultante est également affichée, ce qui peut être utile dans le cas où une personne a une durée limite. \\
		 
		 La troisième étape pour l'utilisateur est de retrouver la commande ffmpeg qui lui permettra de couper sa vidéo. Avec les chemins absolus vers les vidéos, la commande devenait beaucoup trop grand à afficher, nous avons donc choisi de les masquer dans la commande, mais avons mis en place un bouton permettant de copier la commande complète dans le presse papier de l'utilisateur.

	\part*{Améliorations}
		Actuellement, la visualisation de la vidéo s'adapte en fonction des configurations de l'utilisateurs. Il serait utile d'ajouter un nouveau contrôleur afin de pouvoir se déplacer dans la vidéo sans que cela n'influence les paramètres de coupe. \\
		
		Une autre chose est que les temps utilisés afin de définir le début et la fin de la vidéo vont à la seconde près. Alors qu'une vidéo contient encore plusieurs images par secondes. Cette précision pourrait être améliorée. \\
		
		Certains format de vidéo ne sont pas pris en charge. Les modules Qt Quick sont encore très récents et ceux-ci vont être améliorés à l'avenir.
	
\end{document}
